
\Sec{Doing funny stuff with}{Macro Definitions}
\def\n{\\[1.5em]}
\begin{frame}{~}
\solSetStyle{keywordA}{}%
\begin{focus}
   % fragen ob die befehle allen so grob bekannt sind
\begin{minipage}{.65\linewidth}
\begin{columns}[onlytextwidth,c]
\column<2->{.5\linewidth}
   \blatex{\\newcommand}\n
   \blatex{\\renewcommand}\n
   \blatex{\\providecommand}
\column{.1\linewidth}
\column<3->{.25\linewidth}
   \blatex{\\let}\n
   \blatex{\\def}\n
   \blatex{\\gdef}\n
   \blatex{\\edef}\n
   \blatex{\\xdef}
\end{columns}\end{minipage}%
\end{focus}
% stern danach ja nein einfach erklären
\end{frame}

\begin{frame}
   \begin{focus}
      \onslide<2->{%
         \B{ncmd}{\\newcommand}\B{cmdname}{\\todo}%
         \B{arg}{[1]}%
         \B{body-start}{\{\\textbf\{todo:\}~}\kern-3pt\B{arg-use}{\#1}\B{body-end}{\}}
      }
      \begin{onlyenv}<6->
         \vspace*{6em}\par
         \def\arraystretch{1.2}%
         \begin{tabular}{rcl}
            & \multicolumn{2}{l}{\onslide<7->{\B[morekeywords={todo}]{todo-call}{\\todo}\B{arg-main}{\{}\kern-5pt\B{arg-rest}{fun\}}}} \\[1mm]
            & \onslide<8->{~~\(\mapsto\)~~} & \onslide<8->{\B{2-body-start}{\\textbf\{todo:\}~}\kern-3pt\B{2-arg-use}{fun}}\\
            & \fo{\onslide<9->{~~\(\mapsto\)~~}} & \fo{\onslide<9->{\B{3-body-start}{\\ifmmode\\nfss@text\{\\bfseries}\textcolor{gray}{\ldots}}}\\
            & \onslide<10->{\(\vdots\)} & \\
            &\onslide<11->{~~\(\mapsto\)~~} & \onslide<12->{\DoBox{\textbf{todo:}~fun}}
         \end{tabular}
      \end{onlyenv}
   \end{focus}
   \begin{tikzpicture}[U-Over]
      \onslide<3->{\draw[K] ([yshift=-1mm]cmdname.south west) -- ([yshift=-1mm]cmdname.south east) node[midway,below=1mm] (cmdname-desc) {name};}
      \onslide<4->{
         \draw[Kite-] ([yshift=1mm]arg.north) to[bend left=25] ++(5mm,5mm) node[right] {number of arguments};
      }
      \onslide<5->{
         \draw[K] ([yshift=-1mm]body-start.south west) -- ([yshift=-1mm]body-end.south east) node[midway,below=1mm] (cmdname-desc) {body};
      }
   \end{tikzpicture}
   % Example replacement with arguments
\end{frame}

\begin{frame}
   \begin{focus}
      \onslide<2->{\blatex[deletekeywords={fraction}]{\\newcommand\\fraction[2]\{\\ensuremath\{\\frac\{\#1\}\{\#2\}\}\}}}
      \begin{onlyenv}<3->
         \vspace*{6em}\par
         \def\arraystretch{1.2}%
         \begin{tabular}{rcl}
            & \multicolumn{2}{l}{\onslide<4->{\B{main-call}{\\fraction}\B{arg-a}{4}\kern-.5pt\B{arg-b}{2}\B{extra}{3}}} \\[1mm]
            & \onslide<5->{~~\(\mapsto\)~~} & \onslide<5->{\B{main-expanded}{\\ensuremath\{\\frac\{4\}\{2\}\}3}} \\
            & \onslide<8->{\(\vdots\)} & \\
            &\onslide<9->{~~\(\mapsto\)~~} & \onslide<9->{\DoBox{\ensuremath{\frac{4}{\rule{0pt}{.92\ht\strutbox}2}}3}}
         \end{tabular}
      \end{onlyenv}
   \end{focus}
   \begin{tikzpicture}[U-Over]
      \onslide<6->{
         \draw[-Kite] (arg-a.north) to[out=90,in=-30] ++(-3mm,3mm) node[left] {\T{\#1}};
      }
      \onslide<7->{
         \draw[-Kite] (arg-b.north) to[out=90,in=210] ++(3mm,3mm) node[right] {\T{\#2}};
      }
   \end{tikzpicture}
   % Consumption of arguments (spaces, groups)
\end{frame}

\begin{frame}
   \begin{focus}
      \blatex[deletekeywords={fraction}]{\\newcommand\\fraction[2]\{\\ensuremath\{\\frac\{\#1\}\{\#2\}\}\}}
      \begin{onlyenv}
         \vspace*{6em}\par
         \def\arraystretch{1.2}%
         \begin{tabular}{rcl}
            & \multicolumn{2}{l}{{\B{main-call}{\\fraction}\B{arg-a}{\{42\}}\kern-.5pt\B{arg-b}{\{3\}}}} \\[1mm]
            & {~~\(\mapsto\)~~} & {\B{main-expanded}{\\ensuremath\{\\frac\{42\}\{3\}\}}} \\
            & {\(\vdots\)} & \\
            & {~~\(\mapsto\)~~} & {\DoBox{\ensuremath{\frac{42}{\rule{0pt}{.92\ht\strutbox}3}}}}
         \end{tabular}
      \end{onlyenv}
   \end{focus}
   \begin{tikzpicture}[U-Over]
      \draw[-Kite] (arg-a.north) to[out=90,in=-30] ++(-3mm,3mm) node[left] {\T{\#1}};
      \draw[-Kite] (arg-b.north) to[out=90,in=210] ++(3mm,3mm) node[right] {\T{\#2}};
   \end{tikzpicture}
   % Consumption of arguments (spaces, groups)
\end{frame}

\newsavebox\CodeBox

\begin{frame}[fragile]
\begin{lrbox}\CodeBox
\begin{minipage}{.85\linewidth}
\begin{minted}{latex}
\ifdraft
   \typeout{Draft-Mode!}
\else
   \typeout{Other-Mode!}
\fi
\end{minted}
\end{minipage}
\end{lrbox}
   \begin{focus}
      \blatex[deletekeywords={ifdraft}]{\\newif\\ifdraft}}\\[4mm]%
      \onslide<2->{\begin{tabular}{ccc}
         \blatex{\\drafttrue} & & \blatex{\\draftfalse}
      \end{tabular}}\bigskip\\[5mm]
      \onslide<3->{\usebox\CodeBox}
   \end{focus}
   % conditionals (etoolbox etc. stuff)
\end{frame}



\begin{frame}[fragile]
   \begin{lrbox}\CodeBox
   \begin{minipage}{.85\linewidth}
   \begin{minted}[escapeinside=||]{latex}
   \makeatletter

   \newcommand\my@fancy@ns@version{v1.0.0}
   \newcommand\myversion{|\def\at{\textcolor{gray}{@}}\T{\textbackslash\textbf{my\at fancy\at ns\at version}}|}

   \makeatother
   \end{minted}
   \end{minipage}
   \end{lrbox}
   \begin{focus}
   \usebox\CodeBox
   \end{focus}
      % TODO: explain the @ symbol
\end{frame}

\begin{frame}
   \begin{focus}
      \blatex{\\edef} \qquad\quad \blatex{\\expandafter} \quad\qquad \blatex{\\protect}\bigskip\\
      \onslide<2->{\ldots}
   \end{focus}
   % TODO: es gibt noch sooo viel mehr
   % more stuff (like namespacing with @, expandafter, edef, ...)
\end{frame}

\Learning{Use Macros to Ease Your Workflow!}{At leat some, e.g., for names, to reduce repetition\ldots}
% # structured texing