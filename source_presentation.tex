\errorcontextlines99999
\documentclass[aspectratio=169,usepdftitle=true,main=english]{beamer}

\usepackage{xcolor-material}

\colorlet{black}{MaterialBlack}
\colorlet{gray}{MaterialGrey}
\colorlet{red}{MaterialRed}
\colorlet{purple}{MaterialPurple}
\colorlet{blue}{MaterialBlue}
\colorlet{cyan}{MaterialCyan}
\colorlet{green}{MaterialGreen}
\colorlet{orange}{MaterialOrange}

\usepackage[T1]{fontenc}
\usepackage[utf8]{inputenc}
\usepackage[english]{babel}
\usepackage{microtype}

\usepackage{etoolbox}
\usepackage{customdice}
\usepackage{booktabs}

\makeatletter

\usepackage{color-palettes}
\usepackage{fancyqr}
\usepackage[fontsize=14]{beamer-latex-pdfpc-notes}
\usepackage[cpalette,encoding,defaultfont,fakeminted]{sopra-listings}
\solLoadLanguage{fancytex,R}
\solsetmintedstyle{plain number}

\usepackage{beamerthemedividing-lines}
\SetColorProfile{42, 42, 42}{249, 166, 2}{137,64,75}
\definecolor{btdl@color@background}{rgb}{252, 252, 252}

\usepackage[backend=bibtex8,style=numeric]{biblatex}
\usepackage{csquotes}
\usepackage{tabularx,array}
\let\say\enquote
% \addbibresource{references.bib}
% \outro{\printBibCommand}


\usepackage{tikz}
\usetikzlibrary{arrows.meta,backgrounds,shapes.symbols,decorations.pathreplacing,3d,graphs,shapes.multipart}
\usepackage{forest}
\usepackage{siunitx}
\usepackage[glows]{tikzpingus}
\newsavebox\OutPingu
\setbox\OutPingu=\hbox{\tikz{\pingu[lightsaber right,right wing wave, glow,devil wings=btdl@color@primary!80!black!96!purple,wool hat=darkgray,laptop left,eyes wink,blush,heart,body type=legacy,bow tie=pingu@blue]}}
\PostPage{\begin{tikzpicture}[@O]
   \node at(current page.center) {\usebox\OutPingu};
\end{tikzpicture}}

\setbeamerfont{subtitle}{size=\fontsize{19}{19},series=\sbfamily}

% too short of a presentation
\sectionbannerfalse

\usepackage{fontawesome}

\tikzset{
    path image shift/.style={},
    path image/.style={path picture={\node at ([path image shift]path picture bounding box.center) {#1};}},
    U/.style={line cap=round, line join=round}
}

% TODO \temporal argument
% \setbeamercovered{transparent=5}

\def\PlaceFramenumberUpperLeft{}
\def\TitlepageSlideAlign{c}
\def\titlepage{%
\begin{tikzpicture}[@O]
   \node[align=left,font=\usebeamerfont{title}\usebeamercolor{title}] at (current page.center) {%
      \MakeUppercase{\inserttitle}\endgraf
   };
   % \node[below left] at(current page.north east) {\color{gray}\href{https://github.com/EagleoutIce/ekg-cs-presentation}{\faGithub}};
\end{tikzpicture}
}

\tikzset{
   T/.style={text=gray,font=\scriptsize\sffamily},
   @O/.style={overlay,remember picture}
}
\errorcontextlines9999
\lstset{add to literate=
   {<-}{{{\!\(\leftarrow\)\!}}}2
}


\def\CurrentSecText{Doing funny stuff with}
\def\Sec#1#2{\def\CurrentSecText{#1}\section{#2}}
\AtBeginSection{{%
% \setbeamercolor{background canvas}{bg=gray}
\begin{frame}
   \begin{tikzpicture}[@O]
      \node[font=\huge\bfseries,above right=1cm] (@) at (current page.south west) {\insertsection};
      \node[above right,yshift=-2mm, gray,font=\scriptsize\sbfamily] at(@.north west) {\CurrentSecText};
   \end{tikzpicture}
\end{frame}
}}

\newcounter{Learnings}
\def\Learning#1#2{\refstepcounter{Learnings}%
\begin{frame}
   \begin{tikzpicture}[@O]
      \node[gray,font=\scriptsize\sbfamily] (@) at(current page.center) {Learning \theLearnings};
      \node[font=\huge\bfseries,below] at (@.south) {\only<2->{#1}};
      \node[font=\scriptsize,gray,above right=2mm,text width=.42\linewidth,align=flush left] at(current page.south west) {\only<3->{#2}};
   \end{tikzpicture}
\end{frame}
}

\setbeamertemplate{footline}{%
\begin{tikzpicture}[@O]
   \node[above left=1mm,font=\tiny\sbfamily,yshift=2pt,xshift=-.33pt] at(current page.south east) {\strut\thepage\strut};
\end{tikzpicture}%
}

\usepackage{droidsansmono}
\makeatletter
\solSetStyle{basic}{\sol@ttfamily}%
\solSetStyle{keywordA}{\bfseries}% TODO: sbfamily?
\solSetStyle{keywordB}{\color{darkgray}}
\solSetStyle{keywordC}{\color{darkgray}}
\solSetStyle{numbers}{\color{darkgray}}
% \soldisablenumhl

\usetikzlibrary{tikzmark,shadows}
\newcommand*\B[3][]{\tikzmarknode{#2}{\strut\blatex[#1]{#3}\strut}}
\def\DoBox#1{\tikz[baseline=(@.base)]{\node[rounded corners=2pt,fill=white,drop shadow={shadow xshift=.33ex,shadow yshift=-.33ex,opacity=.25,fill=gray}](@){#1};}}

\newsavebox\FocusBox
\newif\iffocusC \focusCtrue
\NewEnviron{focus}{%
   \setbox\FocusBox=\hbox{\parbox\linewidth{\iffocusC\begin{center}\BODY\end{center}\else\BODY\fi}}%
   \begin{layout-full}\begin{tikzpicture}[@O]
      \node at(current page.center) {\usebox\FocusBox};
   \end{tikzpicture}\end{layout-full}%
}
\def\fo#1{\begin{colormixin}{30!white}#1\end{colormixin}}

\usepackage{mathastext}
\outro{\label{LAST}}

\setbeamerfont{section in toc}{size=\fontsize{10}{12},series=\bfseries}
\setbeamertemplate{section in toc}{\strut\foreach\i in{1,...,\inserttocsectionnumber}{\hspace*{3.5em}}\textcolor{gray}{\inserttocsectionnumber.}\space\inserttocsection\bigskip}

\tikzset{
   K/.style={decoration={brace,mirror},semithick,decorate},
   KO/.style={decoration={brace},semithick,decorate},
   U-Over/.style={@O,gray,line cap=round,font=\small}
}

\title{Love of doing}

\let\oldtitlepage\DoTitlepage
\def\DoTitlepage{}

\input{data/title}

\makeatletter
% \def\btdm@inserttotalframenumber{23}%
\begin{document}
\begin{frame}[plain]
   \begin{focus}
      \onslide<2->{\textasciitilde}
   \end{focus}
\end{frame}

\begin{frame}[plain]
%
\definecolor{uulmaccent}{HTML}{A9A28D}
\definecolor{red}{HTML}{A32638}
\begin{tikzpicture}[overlay,remember picture]

\node at (current page.center) {\includegraphics[width=\paperwidth]{img/rocket.png}};

\fill[uulmaccent] (current page.south west)
   rectangle ([xshift=-2cm,yshift=5mm]current page.south east) node[midway,centered] {[Titel einfügen]};

\fill[red] (current page.south east) rectangle ++(-1.85cm,5mm) coordinate (@);

\node[below right,white] at(@) {\kern-6pt\scriptsize \thepage\;/\;\pageref{LAST}};
\node[below,white] at([yshift=5mm]current page.south) {\scriptsize[Datum einfügen]};

\node[below left=8.5mm,yshift=-7mm,xshift=3.5mm,rotate=-50,starburst,drop shadow,fill=white,fill opacity=.95,draw=red,very thick,text=red,inner sep=2mm,scale=1.33] at (current page.north east) {\textbf{New!}};

\node[below=2cm,text width=\paperwidth,align=right,font=\huge\bfseries\color{white},fill=red,inner sep=2mm] at (current page.center) {Titel einfügen~~};

\end{tikzpicture}


   % powerpoint gag offset page number
   % weird image in the middle, fake animation

\end{frame}

\begin{frame}
% \hspace*{4em}
\begin{focus}
\vspace*{2em}\par
\hspace*{-.5em}\begin{minipage}{.8\linewidth}
   \tableofcontents
\end{minipage}
\end{focus}
\end{frame}

% \begin{frame}
% \begin{layout-full}
%    \resizebox\linewidth!{\TitleImageLoad\relax}%
% \end{layout-full}
% \end{frame}

\Sec{Doing funny stuff with}{Macro Definitions}
\def\n{\\[1.5em]}
\begin{frame}{~}
\solSetStyle{keywordA}{}%
\begin{focus}
   % fragen ob die befehle allen so grob bekannt sind
\begin{minipage}{.65\linewidth}
\begin{columns}[onlytextwidth,c]
\column<2->{.5\linewidth}
   \blatex{\\newcommand}\n
   \blatex{\\renewcommand}\n
   \blatex{\\providecommand}
\column{.1\linewidth}
\column<3->{.25\linewidth}
   \blatex{\\let}\n
   \blatex{\\def}\n
   \blatex{\\gdef}\n
   \blatex{\\edef}\n
   \blatex{\\xdef}
\end{columns}\end{minipage}%
\end{focus}
% stern danach ja nein einfach erklären
\end{frame}

\begin{frame}
   \begin{focus}
      \onslide<2->{%
         \B{ncmd}{\\newcommand}\B{cmdname}{\\todo}%
         \B{arg}{[1]}%
         \B{body-start}{\{\\textbf\{todo:\}~}\kern-3pt\B{arg-use}{\#1}\B{body-end}{\}}
      }
      \begin{onlyenv}<6->
         \vspace*{6em}\par
         \def\arraystretch{1.2}%
         \begin{tabular}{rcl}
            & \multicolumn{2}{l}{\onslide<7->{\B[morekeywords={todo}]{todo-call}{\\todo}\B{arg-main}{\{}\kern-5pt\B{arg-rest}{fun\}}}} \\[1mm]
            & \onslide<8->{~~\(\mapsto\)~~} & \onslide<8->{\B{2-body-start}{\\textbf\{todo:\}~}\kern-3pt\B{2-arg-use}{fun}}\\
            & \fo{\onslide<9->{~~\(\mapsto\)~~}} & \fo{\onslide<9->{\B{3-body-start}{\\ifmmode\\nfss@text\{\\bfseries}\textcolor{gray}{\ldots}}}\\
            & \onslide<10->{\(\vdots\)} & \\
            &\onslide<11->{~~\(\mapsto\)~~} & \onslide<12->{\DoBox{\textbf{todo:}~fun}}
         \end{tabular}
      \end{onlyenv}
   \end{focus}
   \begin{tikzpicture}[U-Over]
      \onslide<3->{\draw[K] ([yshift=-1mm]cmdname.south west) -- ([yshift=-1mm]cmdname.south east) node[midway,below=1mm] (cmdname-desc) {name};}
      \onslide<4->{
         \draw[Kite-] ([yshift=1mm]arg.north) to[bend left=25] ++(5mm,5mm) node[right] {number of arguments};
      }
      \onslide<5->{
         \draw[K] ([yshift=-1mm]body-start.south west) -- ([yshift=-1mm]body-end.south east) node[midway,below=1mm] (cmdname-desc) {body};
      }
   \end{tikzpicture}
   % Example replacement with arguments
\end{frame}


\begin{frame}
   \begin{focus}
      \onslide<2->{\blatex[deletekeywords={fraction}]{\\newcommand\\fraction[2]\{\\ensuremath\{\\frac\{\#1\}\{\#2\}\}\}}}
      \begin{onlyenv}<3->
         \vspace*{6em}\par
         \def\arraystretch{1.2}%
         \begin{tabular}{rcl}
            & \multicolumn{2}{l}{\onslide<4->{\B{main-call}{\\fraction}\B{arg-a}{4}\kern-.5pt\B{arg-b}{2}\B{extra}{3}}} \\[1mm]
            & \onslide<5->{~~\(\mapsto\)~~} & \onslide<5->{\B{main-expanded}{\\ensuremath\{\\frac\{4\}\{2\}\}3}} \\
            & \onslide<8->{\(\vdots\)} & \\
            &\onslide<9->{~~\(\mapsto\)~~} & \onslide<9->{\DoBox{\ensuremath{\frac{4}{\rule{0pt}{.92\ht\strutbox}2}}3}}
         \end{tabular}
      \end{onlyenv}
   \end{focus}
   \begin{tikzpicture}[U-Over]
      \onslide<6->{
         \draw[-Kite] (arg-a.north) to[out=90,in=-30] ++(-3mm,3mm) node[left] {\T{\#1}};
      }
      \onslide<7->{
         \draw[-Kite] (arg-b.north) to[out=90,in=210] ++(3mm,3mm) node[right] {\T{\#2}};
      }
   \end{tikzpicture}
   % Consumption of arguments (spaces, groups)
\end{frame}

\begin{frame}
   \begin{focus}
      \onslide<2->{\blatex[deletekeywords={fraction}]{\\newcommand\\fraction[2]\{\\ensuremath\{\\frac\{\#1\}\{\#2\}\}\}}}
      \begin{onlyenv}<3->
         \vspace*{6em}\par
         \def\arraystretch{1.2}%
         \begin{tabular}{rcl}
            & \multicolumn{2}{l}{\onslide<4->{\B{main-call}{\\fraction}\B{arg-a}{\{42\}}\kern-.5pt\B{arg-b}{\{3\}}}} \\[1mm]
            & \onslide<5->{~~\(\mapsto\)~~} & \onslide<5->{\B{main-expanded}{\\ensuremath\{\\frac\{42\}\{3\}\}}} \\
            & \onslide<8->{\(\vdots\)} & \\
            &\onslide<9->{~~\(\mapsto\)~~} & \onslide<9->{\DoBox{\ensuremath{\frac{42}{\rule{0pt}{.92\ht\strutbox}3}}}}
         \end{tabular}
      \end{onlyenv}
   \end{focus}
   \begin{tikzpicture}[U-Over]
      \onslide<6->{
         \draw[-Kite] (arg-a.north) to[out=90,in=-30] ++(-3mm,3mm) node[left] {\T{\#1}};
      }
      \onslide<7->{
         \draw[-Kite] (arg-b.north) to[out=90,in=210] ++(3mm,3mm) node[right] {\T{\#2}};
      }
   \end{tikzpicture}
   % Consumption of arguments (spaces, groups)
\end{frame}

\newsavebox\CodeBox

\begin{frame}[fragile]
\begin{lrbox}\CodeBox
\begin{minipage}{.85\linewidth}
\begin{minted}{latex}
\ifdraft
   \typeout{Draft-Mode!}
\else
   \typeout{Other-Mode!}
\fi
\end{minted}
\end{minipage}
\end{lrbox}
   \begin{focus}
      \onslide<2->{\blatex[deletekeywords={ifdraft}]{\\newif\\ifdraft}}\\[4mm]%
      \onslide<2->{\begin{tabular}{ccc}
         \blatex{\\drafttrue} & & \blatex{\\draftfalse}
      \end{tabular}}\bigskip\\[5mm]
      \onslide<3->{\usebox\CodeBox}
   \end{focus}
   % conditionals (etoolbox etc. stuff)
\end{frame}



\begin{frame}[fragile]
   \begin{lrbox}\CodeBox
   \begin{minipage}{.85\linewidth}
   \begin{minted}{latex}
   \makeatletter

   \newcommand\my@fancy@ns@version{v1.0.0}
   \newcommand\myversion{\my@fancy@ns@version}

   \makeatother
   \end{minted}
   \end{minipage}
   \end{lrbox}
   \begin{focus}
   \onslide<2->{\usebox\CodeBox}
   \end{focus}
      % TODO: explain the @ symbol
\end{frame}

\begin{frame}
   \begin{focus}
      \onslide<2->{\blatex{\\edef} \qquad\quad \blatex{\\expandafter} \quad\qquad \blatex{\\protect}}\bigskip\\
      \ldots
   \end{focus}
   % TODO: es gibt noch sooo viel mehr
   % more stuff (like namespacing with @, expandafter, edef, ...)
\end{frame}

\Learning{Use Macros to Ease Your Workflow!}{At leat some, e.g., for names, to reduce repetition\ldots}
% # structured texing

\Sec{Using other's work by loading}{Important Packages}
{\def\pkg<#1>#2#3{\onslide<#1>{\parbox{12.5em}{\blatex{\\usepackage\{#2\}}} \textcolor{gray}{\small--- #3}}}
\newsavebox\Pengu
\setbox\Pengu=\hbox{\tikz{\pingu[eyes wink,body type=legacy,blush,wings grab,cup,feet sit,:mix-all=15!white]}}
\begin{frame}{}
   \focusCfalse\begin{focus}
      \only<11-|handout:0>{\begin{colormixin}{32!white}}%
      \pkg<2->{microtype}{typographical refinements}\\
      \pkg<3->{booktabs}{nicer looking tables}\\
      \pkg<4->{siunitx}{format your units}\\
      \pkg<5->{csquote}{context sensitive quotation}\bigskip\\
      \pkg<6->{etoolbox}{programming-like constructs}\\
      \pkg<7->{xstring}{powerful string manipulation}\\
      \pkg<8->{xfp}{amazing fixed point arithmetic}\bigskip\\
      \only<11-|handout:0>{\end{colormixin}}%
      \pkg<9->{tikz}{\say{Ti\textit{k}Z ist kein Zeichenprogramm}}\\
      \pkg<10->{forest}{drawing cute trees}
   \end{focus}
   % gibt immer noch viel viel viel mehr
   \begin{tikzpicture}[@O]
      \onslide<12->{
         \node[above right=1mm,scale=.225] (@) at(current page.south west) {\href{https://github.com/EagleoutIce/tikzpingus}{\usebox\Pengu}};
         \node[right=1mm,opacity=.4,scale=.5] (@) at (@.east) {\href{https://github.com/EagleoutIce/tikzpingus}{\blatex{\\usepackage\{tikzpingus\}}  \textcolor{gray}{\small--- for love}}};}
   \end{tikzpicture}
\end{frame}%
}

\Learning{Learn About (New) Packages}{Even \qty{10}{\percent} can save the world\ldots}
% and yes, some journals lock what is available, but often there are alternatives
% tabu deprecated, aber immer und überall neue keine und große Pakete

\Sec{On summoning penguins with}{The Ti\textit{k}Z package}

\begin{frame}{\insertsection}
   tikz absolute placement and nodes
\end{frame}

\Sec{Caring for the nature with}{The Forest Package}
\begin{frame}{\insertsection}
   forests
\end{frame}

\Sec{Understanding some depths with}{Boxes in \LaTeX}
\begin{frame}{\insertsection}
   boxes, llap, rlap, clap, raisebox,
\end{frame}

% \section{Catcodes}
% \begin{frame}{\insertsection}
%    catcodes
%    how does latex expansion work
% \end{frame}

% \section{}
% layouts: port from one venue to another

\begin{frame}
   TODO: summarize
\end{frame}

\appendix
\begin{frame}
   TODO: fun stuff like overset of references etc.
\end{frame}

\begin{frame}
   Non-Breaking Space \textasciitilde,
   Some quick tips like '-' vs. '--' vs. '---'
   open questions round
\end{frame}

\end{document}